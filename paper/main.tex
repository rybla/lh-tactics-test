%% The first command in your LaTeX source must be the \documentclass command.
\documentclass[sigplan,screen,review,anonymous]{acmart}

\newif\ifdraft\drafttrue
%% Notes

\usepackage{xcolor}
\usepackage{xspace}

\definecolor{darkgreen}{RGB}{0,150,0}
\definecolor{darkorange}{RGB}{191,114,13}
\newcommand\todo[1]{\ifdraft\textcolor{red}{\textbf{TODO:} #1}\fi}
\newcommand\nv[1]{\ifdraft\textcolor{purple}{\textbf{NV:} #1}\fi}
\newcommand\mmg[1]{\ifdraft\textcolor{teal}{\textbf{MMG:} #1}\fi}
\newcommand\leo[1]{\ifdraft\textcolor{darkgreen}{\textbf{LEO:} #1}\fi}
\newcommand\hen[1]{\ifdraft\textcolor{darkorange}{\textbf{HENRY:} #1}\fi}

\newcommand{\cn}{\ifdraft\textsuperscript{\textcolor{blue}{[citation needed]}}\xspace\fi}

%% Text 
\newcommand\ie{{i.e.,}\xspace}
\newcommand\eg{{e.g.,}\xspace}
\newcommand\resp{{resp.}\xspace}

%% System names
\newcommand\Fstar{F${}^*$}

%% EDSL names
\newcommand{\LangA}{proof macro language\xspace}
\newcommand{\LangATerm}{proof macro\xspace}
\newcommand{\LangB}{proto-proof language\xspace}
\newcommand{\LangBTerm}{proto-proof term\xspace}

\newcommand{\TheTool}{the tool\xspace}

\usepackage{listings}

% uncomment next line to restore colors
\def\withcolor{}


\ifdefined\withcolor
  \definecolor{fstarblue}{rgb}{0.0, 0.0, 1.0}
  \definecolor{haskellstr}{rgb}{0.2, 0.2, 0.6}
  \definecolor{haskellred}{rgb}{1.0, 0.0, 0.0}
  \definecolor{gray_ulisses}{gray}{0.55}
  \definecolor{castanho_ulisses}{rgb}{0.59,0.42,0.15}
  \definecolor{preto_ulisses}{rgb}{0.55,0.28,0.59}
  \definecolor{green_ulises}{rgb}{0.59,0.42,0.15}
\else
	\definecolor{fstarblue}{gray}{0.1}
	\definecolor{haskellstr}{gray}{0.1}
	\definecolor{haskellred}{gray}{0.1}
	\definecolor{gray_ulisses}{gray}{0.1}
	\definecolor{castanho_ulisses}{gray}{0.1}
	\definecolor{preto_ulisses}{gray}{0.1}
	\definecolor{green_ulisses}{gray}{0.1}
\fi

\def\codesize{\small}

\lstdefinelanguage{HaskellUlisses} {
	basicstyle=\ttfamily\codesize,
	sensitive=true,
	%% morecomment=[s][\color{gray_ulisses}\ttfamily\itshape\codesize]{-}{-},
	%% morecomment=[l][\color{gray_ulisses}\ttfamily\itshape\codesize]{--},
	%% morecomment=[s][\color{gray_ulisses}\ttfamily\itshape\codesize]{\{-}{-\}},
	%% morecomment=[s][]{\{-@}{@-\}},
	morestring=[b]",
	stringstyle=\color{haskellstr},
	basewidth={0.53em},
	showstringspaces=false,
	numberstyle=\codesize,
	numberblanklines=true,
	showspaces=false,
	breaklines=true,
	showtabs=false,
	tabsize=4,
    literate={ {/\\}{{$\land$}}2
             {->}{{$\rightarrow$}}2
			 {<=>}{{$\Leftrightarrow$}}1
%			 {<=}{{$\leq$}}1
%			 {>=}{{$\geq$}}1
             {forall}{{$\forall$}}1
			 {'a}{{$\alpha$}}1
			 {labelty}{{$l$}}1
             {True}{{$\top$}}1
             {~int}{{$\mathbb{Z}$}}1
             {~nat}{{$\mathbb{N}$}}1
			 {==>}{{$\Longrightarrow$}}1
			 {=>}{{$\Rightarrow$}}1
			 {`feq`}{{$\eqinfix$}}1
			 {ka}{{k${}_a$}}1
			 {kb}{{k${}_b$}}1
			 {dollar}{{$\$$}}1
			 {dsl}{{d$_{sl}$}}2
			 {dfs}{{d$_{fs}$}}2
			 {rsl}{{r$_{sl}$}}2
			 {rfs}{{r$_{fs}$}}2
			 {dlm}{{d$_{lm}$}}2
           },
	emph=
	{[1] Set, Level, Axiom, Propositional, Extensionality, Tot, Type, bool, Lemma, ensures, requires, Ifc, IFC, IfcClearance, GlobalInt, GTot
	},
	emphstyle={[1]\color{fstarblue}},
	emph=
	{[2] class, match, with, if, then, else, let, rec, type, val, in, instance, data, measure, where, effect,noeq, private
	},
	emphstyle={[2]\color{castanho_ulisses}},
	emph=
	{[3]
        lattice, value, equals, canFlow, meet, join, bottom, top, 
        lawBot, lawFlowReflexivity, lawFlowAntisymetry, lawFlowTransitivity, 
        lawMeet, lawJoin, labels, 
        lt, lmeet, ljoin, lcanFlow, eq,
        labeled, labeledTCB
	},
	emphstyle={[3]\color{preto_ulisses}\textbf},
	emph=
	{[4]
        Low, Medium, High
	},
	emphstyle={[4]\color{green_ulises}\textbf},
	emph=
	{[5] assume, admit, admitP
	},
	emphstyle=[5]\color{red}\textbf,%\underline, % underline not working
	% this emp 6 is a ban rule => highlight bad code
	emph={[6] leq, equals, join', c_0, c_1
	},
	emphstyle=[6]\color{green}\textbf,
}

\newcommand{\LC}{\lstinline[language=HaskellUlisses, basicstyle=\ttfamily]}

\lstnewenvironment{code}
{\lstset{language=HaskellUlisses}}
{}

\lstnewenvironment{scode}
{\lstset{language=HaskellUlisses,basicstyle=\ttfamily\footnotesize,keepspaces,mathescape}}
{}


\lstnewenvironment{mcode}
{\lstset{language=HaskellUlisses,columns=fullflexible,keepspaces,mathescape}}
{}

\lstnewenvironment{ccode}
{\lstset{language=C,columns=fullflexible,keepspaces,mathescape}}
{}

\lstMakeShortInline[language=HaskellUlisses,mathescape,keepspaces,mathescape,basicstyle=\ttfamily\codesize,breakatwhitespace]|


\usepackage[inference]{semantic}

\newcommand{\Rho}{\mathrm{P}}
\newcommand{\macroHole}[2]{\Box_{#1 ; #2}}
\newcommand{\expandsTo}{\rightsquigarrow}

\newcommand{\Expand}[4]{#1; #2 \vdash #3 \expandsTo #4}
\newcommand{\Interp}[2]{\llbracket #1 \rrbracket (#2)}
\newcommand{\Sem}[3]{\Interp{#1}{#2} = #3}

%% Subfigures
\usepackage{caption}
\usepackage{subcaption}

%% Rights management information.  This information is sent to you
%% when you complete the rights form.  These commands have SAMPLE
%% values in them; it is your responsibility as an author to replace
%% the commands and values with those provided to you when you
%% complete the rights form.
\setcopyright{none}
\copyrightyear{2018}
\acmYear{2018}
\acmDOI{XXXXXXX.XXXXXXX}

%% These commands are for a PROCEEDINGS abstract or paper.
% \acmConference[Conference acronym 'XX]{Make sure to enter the correct
%   conference title from your rights confirmation emai}{June 03--05,
%   2018}{Woodstock, NY}
% \acmPrice{15.00}
% \acmISBN{978-1-4503-XXXX-X/18/06}

%%
%% The majority of ACM publications use numbered citations and
%% references.  The command \citestyle{authoryear} switches to the
%% "author year" style.
%%\citestyle{acmauthoryear}

\begin{document}

\title{Liquid Proof Macros}

\author{Henry Blanchette}
\email{blancheh@umd.edu}
\affiliation{%
  \institution{University of Maryland}
  \city{College Park}
  \country{USA}
}
%\orcid{1234-5678-9012}

\author{Niki Vazou}
\email{niki.vazou@imdea.org}
\affiliation{%
  \institution{IMDEA}
  \city{Madrid}
  \country{Spain}
}

\author{Leonidas Lampropoulos}
\email{leonidas@umd.edu}
\affiliation{%
  \institution{University of Maryland}
  \city{College Park}
  \country{USA}
}

%%
%% By default, the full list of authors will be used in the page
%% headers. Often, this list is too long, and will overlap
%% other information printed in the page headers. This command allows
%% the author to define a more concise list
%% of authors' names for this purpose.
\renewcommand{\shortauthors}{Blanchette et al.}

%%
%% The abstract is a short summary of the work to be presented in the
%% article.
\begin{abstract}
Liquid Haskell is a popular verifier for Haskell programs,
leveraging the power of SMT solvers to ease users' burden of proof.
%
However, this power does not come without a price:
convincing Liquid Haskell that a program is correct
often necessitates giving hints to the underlying solver, which can be
a tedious and verbose process that sometimes requires intricate
knowledge of Liquid Haskell's inner workings.

In this paper, we present {\em Liquid Proof Macros}, an extensible
metaprogramming technique and framework for simplifying the
development of Liquid Haskell proofs.
%
We describe how to leverage Template Haskell to generate Liquid
Haskell proof terms, via a tactic-inspired DSL interface for more
concise and user-friendly proofs,
%
and we demonstrate the capabilities of this framework by automating
a wide variety of proofs from an existing Liquid Haskell benchmark.
\end{abstract}

%%
%% The code below is generated by the tool at http://dl.acm.org/ccs.cfm.
%% Please copy and paste the code instead of the example below.
%%
\begin{CCSXML}
<ccs2012>
   <concept>
       <concept_id>10011007.10011074.10011099.10011692</concept_id>
       <concept_desc>Software and its engineering~Formal software verification</concept_desc>
       <concept_significance>500</concept_significance>
       </concept>
 </ccs2012>
\end{CCSXML}

\ccsdesc[500]{Software and its engineering~Formal software verification}

%%
%% Keywords. The author(s) should pick words that accurately describe
%% the work being presented. Separate the keywords with commas.
\keywords{Liquid Haskell, proof macros, tactics}

%%
%% This command processes the author and affiliation and title
%% information and builds the first part of the formatted document.
\maketitle

\section{Introduction}

Liquid Haskell~\cite{liquidHaskell} is a popular verifier for Haskell
programs, leveraging the power of SMT solvers~\cite{BarST-RR-10} (such
as Z3~\cite{Z3} or CVC4~\cite{CVC4}) to prove the correctness of
diverse applications ranging from optimizations~\cite{TPE2018} to
string matching algorithms~\cite{TaleOfTwoProvers}. Specifications for
these applications are written in the form of {\em refinement
  types}~\cite{RefinementForML}, boolean predicates over program
values.

For concreteness, consider the following \LC{min} function that
computes the minimum of two natural numbers, defined inductively:
\begin{code}
  data N = Z | S N
  
  min :: N -> N -> N 
  min Z _ = Z
  min _ Z = Z
  min (S m) (S n) = S (min m n)
\end{code}

\newcommand{\imin}{\textit{min}~}
Naturally, we would expect such a function to be associative, that is:
$$ \forall a ~ b ~ c. ~ \imin (\imin a~b)~c == \imin a~(\imin~b~c) $$
%
In Liquid Haskell, we can {\em specify} associativity by defining a
refinement type to encode this property, and we can {\em prove}
associativity by defining a term of that type:
%
\begin{code}
  {-@ assoc_min :: a:N -> b:N -> c:N ->
      {_:() | min (min a b) c == min a (min b c)}
  @-}
  assoc_min :: N -> N -> N -> Proof
  assoc_min a b c = ...
\end{code}
%
To Haskell, the type of \LC{assoc_min} is simply a function with three
natural number arguments that returns a \LC{Proof}, which is just a
type synonym for \LC{()}. To Liquid Haskell, however, the type of
\LC{assoc_min} is much more interesting: its return type does not only
specify that the output is a unit, but {\em refines} it so that
associativity of \LC{min} holds for its input arguments. In other
words, the only interesting thing about the result of this function is
its refinement, which constitutes an ``extrinsic style'' proof of
associativity. This is a common enough pattern that Liquid Haskell
supports dropping the ``\LC{_:()}'' part of the refinement for
brevity, as we will also do in the remainder of this paper.

But how does Liquid Haskell decide if the refinement type is true? By
reducing typechecking to verification conditions that SMT solvers
reason about. However, while SMT solvers are pre-programmed with a
wide assortment of facts about various domains such as integer
arithmetic and boolean logic, they don't really know anything about
user-defined data types like \LC{N} or user-defined functions like
\LC{min}. While a direct encoding of such features to SMT is possible
in principle~\cite{HALO}, it unfortunately leads to unpredictable
verification, also known as the ``butterfly
effect''~\cite{LeinoP16}. To that end, Liquid Haskell lifts
user-defined data types and functions into a representation that can
be handled symbolically by SMT solvers~\cite{VazouTCSNWJ18}. Still,
many true properties of user-defined data types and functions are
still not automatically verifiable: users must guide, via refined
Haskell code, the SMT solver to simpler cases that can be checked
automatically.

Unfortunately, given the lack of interactivity of Liquid Haskell, it
is not always clear what the gap in understanding between the user and
the SMT solver is, which often makes writing such refined code a
tedious and frustrating process. Consider again associativity for the
\LC{min} function. On paper, we can informally reason that
associativity holds by induction on the natural numbers that are inputs
to \LC{min}, due to its simple recursive structure. In Liquid Haskell,
the refined code that finally convinces the SMT solver that the
program typechecks is shown in Figure~\ref{fig:assoc-min-proof}.


% The SMT solver, however, does not know about induction (unlike Coq where a tactic "induction" could solve this goal right away).
% And, because of Liquid Haskell's limited capabilities for quantifying over predicates (TODO: include reference here), it is unfeasible to write the induction principle for natural numbers as a refined Haskell function.
% So, a proof of "assoc\_min" is written as follows:
\begin{figure}
\begin{code}
  {-@ assoc_min :: a:N -> b:N -> c:N ->
        {min (min a b) c == min a (min b c)} @-}
  assoc_min :: N -> N -> N -> Proof
  assoc_min a b c =
    case a of 
      Z ->
        case b of 
          Z ->
            case c of
              Z -> trivial
              S c' -> trivial
          S b' ->
            case c of
              Z -> trivial
              S c' -> trivial
    S a' ->
      case b of 
        Z ->
          case c of
            Z -> trivial
            S c' -> trivial
        S b' ->
          case c of
            Z -> trivial
            S c' -> assoc_min a' b' c'
\end{code}
\caption{Liquid Haskell proof term for associativity of \texttt{min}}
\label{fig:assoc-min-proof}
\end{figure}

{\em All} of the branches of pattern matching on \LC{a}, \LC{b}, and
\LC{c} must be written out explicitly. Otherwise, the SMT solver would
not know how to simplify the \LC{min} expressions in the
refinement---the only facts it knows are the three equations that were
used in \LC{min}'s definition:
%
\LC{min Z _ = Z}, \LC{min _ Z = Z}, and
%
\LC{min (S x) (S y) = S (min x y)}.
%
Liquid Haskell understands the constraints introduced by pattern
matching, and takes them into account in order to discharge most
cases---the non-recursive ones that involve at least one \LC{Z}. The
proof conclusion in such cases is \LC{trivial}, which is again just a
synonym for the term-level \LC{()}.

However, in the recursive case of \LC{min}, the Liquid Haskell typechecker needs
additional help, in the form of a recursive call to \LC{assoc_min a' b' c'}, which
brings its refinement in scope for the SMT solver and allows it to conclude that
the induced verification condition holds. Crucially, this refinement is again the
only thing that matters: while the structure of the term gives the appearance
of a proof term in the style of Coq or Agda, the actual return value doesn't matter.
We could just as well have written something like
\begin{code}
  snd (assoc_min a' b' c', ())
\end{code}
and Liquid Haskell would still gladly accept the definition. In fact,
Liquid Haskell's conjunction operator (\LC{&&&}) is defined exactly
this way: it takes two \LC{Proof}s and returns the second one---its
only effect is making the refinement of both arguments visible to the SMT solver.

Even in this simple example of associativity of \LC{min}, the full
verbosity required is cumbersome and obscures the fact that the
underlying argument is a straightforward induction. In larger
developments where the SMT solver might need to rely on helper lemmas,
this problem only becomes more pronounced.  Other proof assistants,
such as Coq~\cite{Coq}, Lean~\cite{Lean4}, or
Isabelle~\cite{Isabelle}, rely on interactive tactics in these
situations to aid users' proof efforts. But developers of these tactic
languages enjoy a transparent API to interact with the current proof
state, and an essentially clean slate to design metaprogramming
capabilities, which has been exploited to the great benefit of proof
assistant users~\cite{Ltac, Mtac, ltac2}.

Unfortunately, Liquid Haskell interacts with the SMT solver in a very
opaque manner, and within the Haskell ecosystem metaprogramming
capabilities are already well established in the form of Template
Haskell---but not designed specifically with SMT-based verification in
mind. So then, {\em what can we do within the confines of this mature
  Haskell ecosystem to aid users?}  Without interactivity, an
interface to concise proof generators must expand to a proof term all
at once i.e. it must behave like a macro. Therefore, we developed a
macro system for generating Liquid Haskell proof terms, using the
existing metaprogramming tools for Haskell.
  % TODO: any other limitations to mention, that directed us to this solution?

\paragraph*{Liquid Proof Macros}

In this paper, we show how to leverage the power of Template Haskell
to automate proof term generation for Liquid Haskell.  We develop {\em
  Liquid Proof Macros}, an extensible DSL in which users can write
intutive proofs that resemble automated tactics%
\footnote{We refrain calling our DSL tactics, as that suggests a
  notion of interactivity that is impossible in the current version
  of Liquid Haskell.}%
%
 of more traditional proof assistants, including case analysis,
 induction, conditioning, and proof search. For example, the same
 proof of associativity of \LC{min} using Liquid Proof macros can be
 seen in Figure~\ref{fig:assoc-min-macro}.

These macros are expanded to a subset of Haskell that resembles, or
rather is even more complicated than, the one used in
Figure~\ref{fig:assoc-min-proof}. To facilitate typechecking of larger
Liquid Haskell developments, we also augment this subset with metadata
information, and provide a pruning algorithm reminiscent of shrinking
in property-based testing~\cite{ClaessenH00}, simplifying away any
unnecessary components that result from proof search.


\begin{figure}[t]
\begin{code}
  {-@ assoc_min :: a:N -> b:N -> c:N ->
        {min (min a b) c == min a (min b c)} @-}
  [tactic|
    assoc_min :: N -> N -> N -> Proof
    assoc_min a b c = induct a; induct b; induct c
  |]
\end{code}
\caption{Associativity of \texttt{min} using Liquid Proof Macros}
\label{fig:assoc-min-macro}
\end{figure}

\pagebreak
In this paper we make the following contributions:
\begin{itemize}
\item We describe a methodology for using Template Haskell to
  automatically construct Liquid Haskell proof terms, and we develop
  an extensible framework using this methodology for automating
  inductive proofs in Liquid Haskell
  (Section~\ref{sec:liquid-proof-macros}).
\item We evaluate our framework against an existing benchmark
  containing a wide variety of Liquid Haskell properties, and found
  that our Liquid Proof Macros can be used to automate all of these
  properties, leading to a $1.57\times$ reduction in code on average
  (Section~\ref{sec:eval})
\end{itemize}
We then discuss related work (Section~\ref{sec:related}), before
concluding with a discussion of the limitations of our framework and
directions for future work (Section ~\ref{sec:future}).


\section{Implementation}

Organization idea:
\begin{enumerate}
  \item Overview the entire procedure:
  \begin{enumerate}
    \item The user writes a proof macro term in ``tactic'' quasiquotes as their proof, in EDSL1 i.e. the proof macro language
    \item The proof macro term is preprocessed into a ``proto proof term'' in EDSL2 i.e. the metadata-augmented subset of Haskell relevant for extrinsic proofs
    \item The ``proto proof term'' is embedded into Haskell, then spliced the place of the original proof macro. The file is checked.
    \item If the check passes, then the proto proof term is pruned by repeatedly pruning the proto proof term, embedding and splicing it into the place of the original proof macro, and checking if the pruned result still passes.
    \item The resulting proof term is a valid proof as determined by Liquid Haskell.
  \end{enumerate}
  \item Detail the features of EDSL1 and the intended behaviors
  \item Briefly describe the details of EDSL2
  \begin{enumerate}
    \item Describe how each feature in EDSL1 is implemented by a preprocessing transformation into EDSL2
    \item Preprocessing EDSL1 into EDSL2 is a contextual transformation, since it needs to keep track of an environment which includes various information (see Language.Core.Syntax.Environment'')
    \item How the ``auto'' term in EDSL2 includes metadata for which terms have have been pruned and which have been determined necessary.
  \end{enumerate}
  \item Detail pruning procedure
  \begin{enumerate}
    \item Clearly, this can be done more efficiently.
    \item Point out further possible work like checking for well-typed applications where the function has refined argument types.
  \end{enumerate}
\end{enumerate}

% 
% * real start
% 
  
The proof macro system processes an input proof macro into an output Liquid Haskell proof term into stages:
\begin{enumerate}
  \item 
  The user writes the input proof macro in Template Haskell quasiquotes. 
  Then the user runs \TheTool \todo{how to talk about this executable} on the file containing the target proof macro to process. 
  The quasiquoted proof macro is parsed into \LangA.
  \item
  The \LangATerm is processed into a corresponding \LangBTerm, and all metadata values are given initial defaults.
  \item
  The \LangBTerm is cached, embedded into Haskell, and then spliced in place of the original proof macro. 
  \item 
  The cached \LangBTerm is repeatedly pruned, using metadata to track pruning progress, where each pruning step involves removing some proof terms contained inside of the entire \LangBTerm, embedding and splicing it in place of the original proof macro, and then running Liquid Haskell to check that the prune step was valid.
\end{enumerate}
  
\subsection{The \LangA}

The \LangA defines a collection of proof macros that are meant to resemble Coq tactics.
The proof macro system is designed to be extendible, so that new proof macros can be added easily by adding a new constructor to the \LangA and then handling the new case for preprocessing.
The syntax for \LangA is the following:

\begin{align*}
  \textit{decl-macro} ::= &
    f ~ : ~ \textut{typ} \\ &
    f ~ \overline{y_i} ~ = ~ \overline{\textit{exp-macro} ;}
  \\[1em]
  \textit{exp-macro} ::= &
    \TC{induct} ~ x \\ | &
    \TC{destruct} ~ \textit{exp} \\ | &
    \TC{assert} ~ \textit{exp} \\ | &
    \TC{dismiss} ~ \textit{exp} \\ | &
    \TC{condition} ~ \textit{exp} \\ | &
    \TC{auto} ~ [\overline{\textit{x} ,}] ~ n \\ | &
    \TC{use} ~ \textit{exp} \\ | &
    \TC{trivial}
  \\[1em]
  \textit{exp} ::= \text{Haskell expression} \\
  \textit{typ} ::= \text{Haskell type (monomorphic)} \\
  \textit{f, x, y_i} ::= \textit{Haskell name} \\
  \textit{n} \in \mathbb{N}
\end{align*}

There are two main types of proof macros: 
\begin{itemize}
  \item
  \textit{Control flow} macros are processed into control flow structures, such as pattern matching. 
  If such a macro yields a control flow structure that has mutliple branches, then a macro-processing branch is created for each of these branches, and the sequence of proof macros following it are processed in each of these branches.
  The control flow macros are:
  \begin{itemize}
    \item
    $\TC{induct} ~ x$ --- patterm matches on $x$, which must be a function argument in the declaration macro.
    In each case of the pattern match, the introduced variables are included in the recursion context at the corresponding argument position of $x$.
    This recursion context keeps track of what expressions are allowed to be given as an argument, in that argument position, to a recursion.
    A macro-processing branch is created for each case.
    Note that the hints and height arguments to this macro are optional, and take values \code{[]} and \TC{3} respectively by default.
    \item $\TC{destruct} ~ \textit{exp}$ --- pattern matches on $\textit{exp}$. A macro-processing branch is created for each case.
    \item $\TC{condition} ~ \textit{exp}$ --- conditions on $\text{exp}$. A macro-processing branch is created for the \TC{then} and \TC{else} branches respectively.
    \item $\TC{assert} ~ \textit{exp}$ --- conditions on $\text{exp}$. A macro-processing branch is created for the \TC{then} branch, but the \TC{else} branch is only filled with \TC{trivial}.
    \item $\TC{dismiss} ~ \textit{exp}$ --- conditions on $\text{exp}$. A macro-processing branch is created for the \TC{else} branch, but the \TC{then} branch is only filled with \TC{trivial}.
  \end{itemize}
  \item
  \todo{think of better name? perhaps \textit{evidence-providing}?}
  \textit{Evidence} macros are processed into terms that provide evidence to the Liquid Haskell checker, such as introducing a lemma to the refinement context.
  \todo{is this the right place to put this note?}
  Note that if a sequence of macros given in declaration macro (i.e. \textit{decl-macro}) does \textit{not} end in an evidence macro, then an \TC{auto} macro is included implicitly at the end of the sequence.
  The evidence macros are:
  \begin{itemize}
    \item
    $\TC{auto} ~ [\overline{\textit{x} ,}] ~ n$ --- generates all well-typed neutral forms that have type \TC{Proof}, up to height $n$, using variables in context and given as hints. 
    The height of a neutral form is the height of its AST, where applications are multi-ary rather than nested binary (e.g. \TC{f x y z} has height 2 rather than height 4).
    Recursive neutral forms are also generated in this way, but have the additional restriction that a recursive neutral form must have as one of its arguments a variable from that argument position's recursion context, ensuring that Liquid Haskell will be able to determine that the resulting Haskell function is terminating.
    \item $\TC{use} ~ \textit{exp}$ --- includes the refinement of the expression's type into the refinement context.
    \item $\TC{trivial}$ --- includes \TC{trivial :: Proof} into the resulting Haskell term. Note that \TC{trivial = ()} and \TC{Proof = ()}.
  \end{itemize}
\end{itemize}

For example, the following is a proof macro for generating a proof that
if \LC{x} is in a list \LC{xs}, then \LC{x} is also in the list \LC{xs ++ ys}:
\begin{code}
  {-@ 
  elem_concat ::
    x:N -> xs:[N] -> ys:[N] ->
    {elem x ys ==> elem x (xs ++ ys)}
  @-}
  [tactic|
  elem_concat :: N -> [N] -> [N] -> Proof
  elem_concat x xs ys =
    induct xs;
    condition {elem x ys}
  |]
\end{code}


% TODO: put this in the implementation details section:
Note that, for Liquid Haskell, the refinement of the type of \TC{f x y, trivial)} is the same as the refinement of the type of \TC{f x y}.
So, the refinement of the expression's type can be included into the refinement context by simply adding it into the resulting Haskell expression

 The proof of "assoc\_min" can be expressed shortly as follows:
  
 The "induct a" macro does case analsis on "a", resulting in two branches. Since both branches are to be handled, the following two macros are executed in both branches (matching the behavior of the ";" tactic combinator in Coq).
 Additionally, in the "S a'" case, the "induct a" tactic notices that "a'" was introduced by inducting on an input variable term in positon 0, so "a'" is marked as a valid argument to a recursive call to "assoc\_min" in position 0.
 The same happens for "induct b" and "induct c", yielding the same branches of the verbose Haskell term.
 Finally, every proof macro implicitly ends with "auto" unless it ends with "trivial" or "use e".
 So, in each of the 9 branches, "auto" is executed to generate all well-typed terms of type "Proof" using things in local context.
 In all but the "S a', S b', S c'" case, there are no such well-typed applications, since the only way to produce a term of type "Proof" is to use "assoc\_min", but only in that final is there a term in context that is available to use as the argument to a recursive call to "assoc\_min", for each of its argment positions (which are 0, 1, 2 since it has 3 arguments and each is inducted on).
  
\subsection{Preprocessing}
  
%s EDSL1 is used by the user to write proof macros inside of quasiquotes of the form \texttt{[tactic|...|]}
%  
% %Grammar of EDSL1:
% \begin{align*}
%   \textit{decl}~ ::= &
%     \textit{name} ~ : ~ \textit{type} \\ &
%     \textit{name} ~ \overline{\textit{name}} ~ = ~ \overline{\textit{instr} ; }
%   \\
%    \textit{instr} ~ ::= &
%      \textbf{intro} ~ \textit{name} \\ | &
%      \textbf{destruct} ~ \textit{name} ~ (\textbf{as} ~ \textit{destruct-pat}) ([\overline{\textit{flag},}]) \\ | &
%      \textbf{induct} ~ \textit{name} ~ (\textbf{as} ~ \textit{pat}) ([\overline{\textit{induct-flag},}]) \\ | &
%      \textbf{auto} ~ (\overline{\textit{name},}) (\textit{nat}) \\ | &
%      \textbf{condition} ~ \textit{exp} ~ (\textbf{requires} ~ [\overline{\textit{name},}]) \\ | &
%      \textbf{assert} ~ \textit{exp} ~ (\textbf{requires} ~ [\overline{\textit{name},}]) \\ | &
%      \textbf{dismiss} ~ \textit{exp} ~ (\textbf{requires} ~ [\overline{\textit{name},}]) \\ | &
%      \textbf{use} ~ \textit{exp} ~ (\textbf{requires} ~ [\overline{\textit{name},}]) \\ | &
%      \textbf{trivial}
%  \end{align*}
%   
 The language EDSL1, defined above, is preprocessed into EDSL2 which is a subset of Haskell augmented with metadata. 
 The conceptual distinction between EDSL1 and EDSL2 is that preprocessing EDSL1 requires a context and producing many expressions per single EDSL1 instruction, whereas EDSL2 is strictly an embedding into Haskell and so each EDSL2 expression corresponds to exactly one Haskell expression.
  
% Grammar of EDSL2:
% \begin{align*}
%   \textit{decl}~ ::= &
%     \textit{name} ~ : ~ \textit{type} \\ &
%     \textit{name} ~ \overline{\textit{name}} ~ = ~ \textit{expr}
%   \\
%   \textit{expr}~ ::= &
%     \mathbf{\lambda} ~ \textit{name} ~ \textbf{->} ~ \textit{instr} \\ &
%     \textbf{case} ~ \textit{exp} ~ \textbf{of} ~ \overline{\textit{pat} ~ \textbf{->} ~ \textit{exp} ;} \\ & 
%     \textbf{if} ~ \textit{exp} ~ \textbf{then} ~ \textit{exp} ~ \textbf{else} ~ \textit{exp} \\ & 
%     \textbf{auto} ~ [\overline{\textit{exp} ;}] \textit{pruning-metadata} \\ & 
%     \textbf{trivial}
% \end{align*}
%  
 An EDSL1 declaration is preprocessed into an EDSL2 declaration like so:
  
 \begin{verbatim}
 preprocessDecl : EDSL1-decl -> EDSL2-decl
 preprocessDecl (name, args, type, instrs) =
   set function name to name
   set output type to type
   for each arg
     get output type as a -> b 
     store type of arg as a
     set output type to b
   preprocess instrs
  
 // stateful
 preprocess : [EDSL1-instr] -> EDSL2-expr
 preprocess [] = []
 preprocess (instr : instrs) = case instr of 
   intro x -> do
     [|a <- b|] <- get output type
     type of x := a
     output type := b
     [|\x -> $(preprocess instrs)|]
   destruct exp requires xs -> do
     if all of xs are in scope then 
       varss <- get patterns for deconstructing exp
       constrs <- constructors of the datatype of exp
       matches <- 
         for each (vars, constr) in (varss * constrs)
           args <- argument types of constr
           for each (var, arg) in in (pats * args)
             type of var := arg
           [|ps -> $(preprocess instrs)|]
       [|case exp of matches|]
     else
       preprocess instrs
   induct exp requires xs -> do
     if all of xs are in scope then 
       varss <- get patterns for deconstructing exp
       constrs <- constructors of the datatype of exp
       matches <-
         for each (vars, constr) in (varss * constrs)
           args <- argument types of constr
           for each (var, arg) in in (pats * args)
             if exp is an argument to the top-level proposition then
               add var to context of recursive-safe expression to fill the
               respective argument of a recursive call
             type of var := arg
           [|ps -> $(preprocess instrs)|]
       [|case exp of matches|]
     else
       preprocess instrs
   auto hints depth ->
     generate all neutral forms in context, up to depth
     when a recursion is generated, at least one of its arguments must be something from the context of recursive safe expressions for that argument (which is populated via induct)
   assert exp requires xs ->
     if all of xs are in scope then
       [|if exp then $(preprocess instrs) else True|]
     else
       preprocess instrs
   dismiss exp requires xs ->
     if all of xs are in scope then
       [|if exp then True else $(preprocess instrs)|]
     else
       preprocess instrs
   use exp requires xs -> 
     [|use exp &&& $(preprocess instrs)]
   condition exp requires xs ->
     [|if exp then $(preprocess instrs) else $(preprocess instrs)]
   trivial ->
     [|trivial|]
 \end{verbatim}

 %  TODO: use running example, simpler than min_assoc -- something that uses induct, auto (with lemma hints), cond

Consider the following example proof macro:
% {-@ reflect prop @-}
% prop x xs ys = 
%   if elemListN x ys then 
%     elemListN x (concatListN xs ys)
%   else 
%     True
%  
% {-@ automatic-instances proof @-}
% {-@
% proof :: x:N -> xs:ListN -> ys:ListN -> {prop x xs ys}
% @-}
% -- elemListN x1 (concatListN (Cons x2 xs) ys)
% -- elemListN x1 (Cons x2 (concatListN xs ys))
% -- if x1 == x2 then 
% --   QED
% -- else
% --   elemListN x1 (concatListN xs ys)
% -- [tactic|
% -- proof :: N -> ListN -> ListN -> Proof
% -- proof x xs ys =
% --   induct xs;
% --   condition {elemListN x ys}
% -- |]
% -- %tactic:begin:proof
% proof :: N -> ListN -> ListN -> Proof
% proof = \x -> \xs -> \ys -> case xs of
%                                 Data.Nil -> if elemListN x ys then trivial else trivial
%                                 Data.Cons n_0 listN_1 -> if elemListN x ys
%                                                           then proof x listN_1 ys
%                                                           else trivial
The following procedure describes how \LC{elem_concat} is processed into \LangB:
\begin{enumerate}
  \item 
\end{enumerate}
  
\subsection{Pruning}

\subsubsection{Linear Pruning}

For each auto site (i.e. each apreprocearance of \textbf{auto} in the preprocessed EDSL2), each \textit{exp} it uses is pruned one at a time, checking to see if Liquid Haskell still passes after the pruning. If a prune succeeds the pruned \textit{exp} is discarded, otherwise the pruned \textit{exp} is returned to the original \textbf{auto} site.


\section{Evaluation}

The benefits we aim to provide with these proof macros are:
- conciseness: the macros cater towards a specific subset of LH that is used for extrinsic-style proofs, so the interface to this subset can be more specific and restricted than generic programs
- reduced redundancy: often, multiple logical branchings can be handled by the same proof macro (due to the modularity the macros achieve in a similar style to tactics), and proof macro branchings (such as destruct, induct, condition) allow the handling proof macro to be written just once and then used in all of the branches
- modularity: in the same style as tactics, the auto proof macro is contextual and so the same use of auto can be used to solve many different proof goals by leveraging contextual information such as the lemmas, variables, and sound recursions.

To verify the applicability of these benefits, we selected an independently-curated collection of properties to be proved in an extrinsic style using Liquid Haskell: https://github.com/mustafahafidi/qc-to-lh. The original use of these properties was to demonstrate the usefulness of another approach to generating Liquid Haskell proofs that has some similarities to our approach.
TODO: describe similarities and differences
The properties are relatively basic properties about the natural numbers, lists of natural numbers or pairs of natural numbers, and trees of natural numbers.


\subsection{Results}

All of the properties were able to be proven using our proof macro tool.
Each property was proven more consisely than if the proof had been written as a general Haskell program.
TODO: how exaclty to measure this (conciseness)?
This was achieved through reduced redundancy and leveraging the macro-ness of having as a target a specific subset of Haskell.=


\section{Related Work}
\label{sec:related}

\paragraph{Meta F*}
\begin{itemize}
\item \cite{MetaF}
\item Closest (refinement types)
\item Same issues (SMT non-interactive)
\item Heavier-weight solution reifying VC state for users (could be really helpful)
\item More flexibility ecosystem wise.
\item Our goal was to simplify user experience within the much more rigid constraints
  of Haskell. In a free-world, the Meta-F* approach is probably better.
\end{itemize}

\paragraph{Interactive Tactics}
\begin{itemize}
\item Related: tactics in interactive proof assistants like Coq, Lean or Isabelle.
\item Coq has seen a ton of work in this, starting with the original
  Ltac implementation.
\item Mtac was a monadic/typed version
\item Ltac2 is a recent revision that is starting to get traction~\cite{ComputingCorrectly}
\item Coqhammer~\cite{CoqHammer, Coqhammer-sauto}
\item In Lean, just tactics?
\item In Isabelle, hammers and tactics.\leo{Which ones?}
\end{itemize}

\paragraph{Liquid Haskell Automation}
\begin{itemize}
\item PLE (complementary) \leo{Which citation?}
\item Master thesis (not intuitive syntax, source of benchmark, never adopted)
\item Rewriting paper? \cite{REST-rewriting}
\end{itemize}




\section{Further Work}

- integrate into Liquid Haskell as a plugin or part of the Liquid Haskell plugin
- handle the splices code implicitly, rather than having to splice the new code in place
  - i.e. make it more like normal Template Haskell splicing
- handle lemma applications where the lemma has refined arguments
  - this currently isn't supported because Template Haskell doesn't know about refinements
- support polymorphism
  - requires a custom unification algorithm because Template Haskell doesn't have \LC{unify}
  - there's a family of extensions related to polymorphism, such as supporting tyepclass constraints, etc.
- support deeper pattern matching
  - currently, descruct and induct are the only two ways to pattern match, and they only pattern match on exactly one level of destruction
- more efficient pruning
  - need to search for an arbitrary subset of the generated lemma applications, so at worst linear search is asymptotically best, BUT usually we are looking for a set of 0-2 lemma applications, so can do better in practice 
- more advanced forms of \LC{auto} tactic
  - \LC{refine e[?1, ..., ?n]} tactic:
    - the expression \LC{e[?1, ..., ?n]} is a neutral form that contains a collection of holes.
    - since \LC{e} is in neutral form, each hole can have its type determined by context
    - then, the same \LC{auto} implementation can be invoked to generate terms to fill these holes, using things in scope
    - the resulting neutral forms can be pruned as usual for results from \LC{auto}
- better auto hints interface
  - basically, start copying features from Coq 
  - pre-defined collections of hints
  - include/exclude hints at top level or per definition
  - allow arbitrary neutral forms as hints -- not just variables

%%
%% The acknowledgments section is defined using the "acks" environment
%% (and NOT an unnumbered section). This ensures the proper
%% identification of the section in the article metadata, and the
%% consistent spelling of the heading.
\begin{acks}
  We thank Jacob Prinz and the anonymous reviewers for their helpful
  comments.  This work was supported by NSF award \#2107206, {\em
    Efficient and Trustworthy Proof Engineering} (any opinions,
  findings and conclusions or recommendations expressed in this
  material are those of the authors and do not necessarily reflect the
  views of the NSF).
\end{acks}

%%
%% The next two lines define the bibliography style to be used, and
%% the bibliography file.
\bibliographystyle{ACM-Reference-Format}
\bibliography{local, leo}

%%
%% If your work has an appendix, this is the place to put it.
% \appendix

% \section{Proofs}

\end{document}
\endinput
